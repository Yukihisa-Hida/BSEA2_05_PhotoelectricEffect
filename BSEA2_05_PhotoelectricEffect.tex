\documentclass{jarticle}

\usepackage[dvipdfmx]{graphicx}
\usepackage{float}
\usepackage{url}

\title{光電効果}
\author{2511198 肥田幸久 \\ 共同実験者 \\ }
\date{2025年10月22日作成}

\begin{document}
\maketitle



\section{実験の目的}

光電管を用いて, 基礎物理定数のひとつであるプランク定数$h$を測定する.
また, 光電効果の諸特性について得られた測定値をもとに検証を行う.



\section{実験の原理}

光電効果とは, 金属表面に特定の振動数以上の光が当たると電子(光電子)が飛び出す現象である.
光電子の運動エネルギー$K$と光の振動数$\nu$, 金属表面から光電子が飛び出すのに必要なエネルギー(仕事関数)$W$には, プランク定数$h$を用いて次式の関係が成り立つことが知られている.
\begin{equation}
  K = h\nu - W
\end{equation}



\section{実験方法}



\section{実験結果}



\section{考察}



\begin{thebibliography}{99}

  \bibitem{金属の密度} \url{https://www.eng-book.com/sample/pdf/P268.pdf}

\end{thebibliography}

\end{document}